\chapter{Conclusioni}
\label{chap:conclusioni}

\section{Consuntivo finale}
Lo stage si è rivelato estremamente utile e proficuo, fornendomi quelle 
competenze pratiche che sentivo di non possedere a causa della natura teorica 
della maggior parte dei corsi previsti dalla laurea.
Anche gli argomenti trattati e le tecnologie utilizzate durante il progetto mi sono sembrate estremamente attuali, quindi sono sicuro che le competenze 
acquisite in questi 2 mesi mi torneranno molto utili in futuro.\\
Mi ritengo estremamente soddisfatto ed orgoglioso dei prototipi sviluppati in questi due mesi in collaborazione con i tutor aziendali, persone che ho 
trovato estremamente disponibili e capaci, e il compagno di stage Endi, 
persona estremamente in gamba ed affidabile. Siamo riusciti a soddisfare tutti 
gli obiettivi individuati per i vari periodi, spesso in anticipo, 
permettendoci di sviluppare funzionalità aggiuntive ed 
esplorare altri aspetti interessanti dell'ambito della \textit{Generative AI}.\\
Entrambe le versioni sviluppate dall'assistente, ovviamente ancora 
migliorabili sotto molti aspetti, sono tuttavia in grado di fornire proposte 
di risoluzione corrette e dettagliate in modo costante, fornendo allo stesso 
tempo una prospettiva sulle prestazioni ottenibili utilizzando modelli locali 
come alternativa a quelli \textit{cloud}.

\section{Raggiungimento degli obiettivi}
Gli obiettivi prefissati nel piano di lavoro sono stati raggiunti interamente, anche se non sempre all'interno dei \textit{sprint} designati.
Questi obbiettivi possono essere sintetizzati nella tabella \ref{tab:internship_scope}.

\setlength{\tabcolsep}{8pt}
\begin{center}
    \rowcolors{1}{}{tableGray}
    \begin{longtable}{|p{12cm}|p{2.5cm}|p{2.25cm}|}
    \hline
    \multicolumn{1}{|c|}{\textbf{Obiettivo}} & \multicolumn{1}{c|}{\textbf{Raggiunto}}\\ 
    \hline 
    \endfirsthead
    \rowcolor{white}
    \multicolumn{3}{c}{{\bfseries \tablename\ \thetable{} -- Continuo della tabella}}\\
    \hline
    \multicolumn{1}{|c|}{\textbf{Obiettivo}} & \multicolumn{1}{c|}{Raggiunto}\\ \hline 
    \endhead
    \hline
    \rowcolor{white}
    \multicolumn{3}{|r|}{{Continua nella prossima pagina...}}\\
    \hline
    \endfoot
    \endlastfoot 
    Gestire l’installazione e l’avvio dell’ambiente comprendente database vettoriale (da decidere), LLM, e eventuali altri componenti necessari. & \cellcolor{emerald}\textcolor{white}{Affermativo}\\
    \hline
    Creazione del modulo LLM locale che sostituisca in modo trasparente AWS Bedrock. & \cellcolor{emerald}\textcolor{white}{Affermativo}\\
    \hline
    Definire procedura di training con il re-import dei dati e creazione di una interfaccia che consenta lo switch da un servizio di generative AI all’altro in modo dinamico senza richiedere il deploy del software & \cellcolor{emerald}\textcolor{white}{Affermativo} \\
    \hline
    Creazione del processo di retroazione per dare feedback al modello di LLM per migliorare dall’apprendimento continuo la qualità delle risposte. & \cellcolor{emerald}\textcolor{white}{Affermativo} \\
    \hline
    Creazione del processo che in modo visuale consenta di mostrare in modo trasparente e semplice agli utenti che il modello è stato addestrato secondo i carismi di trasparenza, sicurezza ed eticità. & \cellcolor{emerald}\textcolor{white}{Affermativo} \\
    \hline
    Definire una procedura che consenta agli operatori di valutare in modo trasparente come è stato addestrato il modello per garantire una trasparenza e obiettività delle risposte, 
    ad esempio indicando sempre i dati che hanno portato alla generazione delle risposte & \cellcolor{emerald}\textcolor{white}{Affermativo} \\
    \hline
    \hiderowcolors
    \caption{Raggiungimento degli obiettivi.}
    \label{tab:internship_scope}
    \end{longtable}
\end{center}

\newpage
\section{Conoscenze acquisite}
Durante il tirocinio ho avuto la possibilità di ampliare enormemente le mie conoscenze informatiche attraverso lo studio e l'applicazione di diverse tecnologie mai utilizzate finora, così come affinare e perfezionare l'uso di altre già incontrate in precedenza. 
Per la prima volta ho programmato in Javascript e in Typescript, imparando l'origine e la relazione che intercorre tra i due linguaggi e la loro versatilità d'uso.\\
Ho imparato la differenza tra database classici e No-Sql attraverso l'uso di MongoDB.
Inoltre ho avuto modo di esplorare a fondo tematiche interessanti come l'Intelligenza Artificiale Generativa e la tecnica della \textit{Retrieval Augmented Generation}, le quali sono al giorno d'oggi tra gli argomenti più attuali e ricercati nel mondo dell'informatica. 
Il loro impatto sulla nostra società può essere osservato ormai nella vita di tutti i giorni.
Ho potuto conoscere ed usare LangChain, uno dei \textit{framework} più utilizzati per lo sviluppo di applicazioni 
in tema AI grazie alla sua estensiva libreria di integrazioni con diversi LLMs e strumenti messi a 
disposizione per semplificare la realizzazione e l'implementazione di funzionalità come la generazione 
di testo, il completamento di frasi, la risposta a domande, la sintesi di documenti, e molto altro.\\
Inoltre, anche se non utilizzate direttamente o in modo estensivo in quanto facenti parte principalmente 
del piano di lavoro del compagno di progetto, ho arricchito il mio bagaglio conoscitivo riguardo il 
mondo del \textit{Cloud Computing} e i servizi offerti da \textit{Amazon Web Services}.\\

Svolgendo lo stage in azienda, ho potuto interfacciarmi con il mondo del lavoro nell'ambito dell'informatica, sviluppando una vasta gamma di \textit{soft skills} che saranno importanti per quando
dovrò entrare a farne parte.\\
Mi ha permesso di entrare in contatto con le dinamiche quotidiane di sviluppo di un applicativo 
software, apprendere l'uso di metodologie agili di sviluppo e lo sviluppo in collaborazione, imparare a gestire task ed obiettivi giornalieri in linea con il piano di lavoro e le complessità di integrazione di sistemi.\\


\section{Valutazione personale}

Al termine di questa esperienza, ritengo sia fondamentale sottolineare l'importanza della pratica nel
percorso formativo di un informatico. Lo studio delle tecnologie, dei linguaggi di programmazione e 
delle metodologie è senza dubbio essenziale, poiché fornisce le basi teoriche necessarie per comprendere 
il funzionamento e le potenzialità degli strumenti che utilizziamo quotidianamente. 
Tuttavia, queste conoscenze teoriche, per quanto approfondite e dettagliate, rischiano di rimanere 
astratte se non vengono applicate in contesti reali. 
La pratica è ciò che completa e solidifica l'apprendimento teorico, trasformandolo in una risorsa concreta e tangibile: la vera acquisizione delle competenze avviene quando si è in gradi di applicare la teoria.

\newpage