% Acronyms
\newacronym{api}{API}{Application Program Interface}
\newacronym{sdk}{SDK}{Software Development Kit}
\newacronym{uml}{UML}{Unified Modeling Language}
\newacronym{rag}{RAG}{Retrival Augmented Generation}


% Glossary
\newglossaryentry{apig}{
    name={API},
    text={Application Program Interface},
    sort=api,
    description={In informatics, an API is a set of procedures available to programmers, typically grouped to form a toolkit for a specific task within a program. Its purpose is to provide an abstraction, usually between hardware and the programmer or between low-level and high-level software, simplifying the programming process}
}

\newglossaryentry{sdkg}{
    name={SDK},
    text={Software Development Kit},
    sort=sdk,
    description={A Software Development Kit (SDK) is a collection of development tools in one installable package, facilitating application creation by providing a compiler, debugger, and sometimes a software framework. SDKs are typically specific to a hardware platform and operating system combination. Many application developers use specific SDKs to enable advanced functionalities such as advertisements, push notifications, etc}
}

\newglossaryentry{umlg}{
    name={UML},
    text={Unified Modeling Language},
    sort=uml,
    description={In software engineering, Unified Modeling Language (UML) is a modeling and specification language based on the object-oriented paradigm. UML serves as a "lingua franca" in the object-oriented design and programming community. Much of the industry literature uses UML to describe analytical and design solutions in a concise and understandable way for a broad audience}
}

\newglossaryentry{TermineSenzaAcronimo}{
    name={Nome del termine},
    sort=termine senza acronimo,
    description={Descrizione}
}

\newglossaryentry{ragg}{
    name={RAG},
    text={Retrival Augmented Generation},
    sort=rag,
    description={Nell'ambito delle Intelligenze Artificaili, la Retrieval-Augmented Generation (RAG) è il processo di ottimizzazione dell'output di un modello linguistico di grandi dimensioni, in modo che faccia riferimento a una base di conoscenza autorevole al di fuori delle sue fonti di dati di addestramento prima di generare una risposta. I modelli linguistici di grandi dimensioni (LLM) vengono addestrati su vasti volumi di dati e utilizzano miliardi di parametri per generare output originali per attività come rispondere a domande, tradurre lingue e completare frasi. La RAG estende le capacità già avanzate degli LLM a domini specifici o alla knowledge base interna di un'organizzazione, il tutto senza la necessità di riaddestrare il modello. È un approccio conveniente per migliorare l'output LLM in modo che rimanga pertinente, accurato e utile in vari contesti.}
}